Le contrôleur est, comme on l'a déjà dit précédemment, l'élément central du réseau SDN, puisqu'il offre au niveau de son interface nord, une API pour développer des applications réseau, et, au niveau de son interface sud, il contrôle les entités réseaux se chargeant du plan de données, avec le protocole Openflow.

Pour fonctionner correctement, le contrôleur doit avoir la représentation interne la plus exacte possible de la topologie réseau qu'il dirige. Pour cela, Openflow prévoit certains paquets spécialisés. Mais ce n'est pas suffisant, puisque les switchs eux-mêmes ne sont pas capables de renseigner le contrôleur sur la topologie alentours. C'est pourquoi certains mécanismes sont mis en place (qui dépendent généralement du contrôleur, même si, devant utiliser des protocoles classiques compréhensibles par des switchs, les possibilités restent limitées).\\
Le mécanisme que j'ai été amené à constater est celui de l'utilisation de paquets LLDP fabriqués par le contrôleur et envoyés aux switchs sous forme de PACKET_OUT. En recevant un tel paquet, un switch va le retransmettre en broadcast aux switchs alentours, qui, normalement, sont configurés pour le renvoyer en PACKET_IN au contrôleur (comportement par défaut, si aucun flux gérant ce type de paquet n'est spécifié, ce qui est préférable). Or, un PACKET_IN encapsule toutes les informations nécessaires au contrôleur pour mettre à jour la topologie locale : en vérifiant que c'est bien lui qui est à l'origine de l'émission du paquet LLDP initial (avec un champ spécial par exemple), il sait que 

Le premier est une actualisation en fonction des PACKET_IN reçus. En effet, ceux-ci sont des encapsulations de paquets réels circulant sur le réseau, on peut donc y lire des adresses ethernet, des adresses IP, .... Le problème consistant bien sûr 