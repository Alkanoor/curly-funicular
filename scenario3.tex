\underline{Prérequis/Hypothèses :}\\
- Un switch malveillant connecté à un switch du réseau ou directement au contrôleur\\
- Une topologie en partie connue (adresse mac et IP des switchs à attaquer)\\


\underline{Buts :}\\
Surcharger les tables de flux d’un ou de plusieurs switch(s) pour provoquer un deni de service (moins de bande passante).

\underline{Déroulement :}\\
Notre switch malveillant peut envoyer à ses switchs voisins des paquets avec une adresse IP source, une adresse MAC source, un VLAN id, un type de service, un port TCP/UDP, aléatoires. Ainsi, les chances que le switch cible envoie un PACKET\_IN au contrôleur sont élevées. D’une part on consomme ainsi des ressources en envoyant beaucoup de PACKET\_IN, et d’autre part si les décisions prises par le contrôleur sont trop spécifiques (peu de jokers utilisés par exemple), le switch sur lequel seront appliquées les règles va progressivement se retrouver surchargé de règles inutiles.

\underline{Détails techniques :}\\
Se référer au scénario 3 décrit dans l'annexe (page).

\underline{Résultat :}\\
L’attaque est réussie : dans les conditions théoriques testées, on passe d’une bande passante de 6,7 Gb/s entre 2 hôtes, à une bande passante de quelques Mb/s. Parfois le déni de service est moins élevé et fournit des variations de débits importantes. Cela s'explique notamment par le fait que certaines règles souvent utilisées restent en haute priorité sur le switch malgré les tentatives de surcharge des tables de flux (et donc sont utilisées sur le plan de données avec des performances acceptables malgré les règles poubelles ajoutées).

\underline{Parades proposées :}\\
Les parades pour cette attaque sont assez nombreuses et relativement faciles à mettre en œuvre. Tout d’abord disposer au niveau du contrôleur d’algorithmes de création de règles capables de rassembler plusieurs règles en une seule (c'est à dire capacité de factoriser des règles avec des jokers). D’autre part une politique de filtrage générale (par exemple DROP des paquets sur certaines IP/pour certains ports ou autre) s’avère très efficace. En résumé, une politique réseau stricte conservant la flexibilité initiale avec des jokers dans les règles ajoutées aux tables de flux.

\underline{Limitations/Impact/probabilité :}\\
Pour cette troisième attaque, l’attaquant doit encore avoir un accès proche du réseau (il doit être connecté à un switch du réseau). De plus, si les switchs sont correctement configurés à la base (admettons que la politique du contrôleur soit une politique "opt-in" et non "opt-out", c’est à dire que par défaut les paquets ne sont pas transmis en PACKET\_IN au contrôleur mais jetés, sauf cas choisis par le contrôleur), alors l’attaque ne fonctionne plus. Openflow à partir de sa version 1.3 permet d’ailleurs à l’administrateur de définir les actions à appliquer à des paquets inconnus (auparavant ils étaient envoyés au contrôleur dans tous les cas). Les switchs sont normalement sensés pouvoir gérer un nombre suffisant de flux et de règles (cela est spécifié dans le premier OFPT\_FEATURES\_REPLY, par exemple avec les switchs virtuels mininet, ce paquet indique le support de 256 tables de flux). Le risque majeur de l’attaque est donc finalement l’écrasement de règles utiles par des règles qui ne le sont pas. Le risque est faible, l’impact moyen.