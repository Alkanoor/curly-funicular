En 2011, l'Open Networking Foundation (ONF) est créée. Regroupant des gros acteurs comme Google, Yahoo, Facebook, Verizon, Microsoft ou encore Deutsche Telekom, c'est l'organisme principal qui encourage l'adoption de la technologie SDN, en publiant régulièrement de nouvelles spécifications Openflow.\\

Google, en 2012, présente, pour la première fois, une architecture SDN pour ses datacenters, utilisant Openflow sur des switchs conçus par eux-mêmes (étant pionniers, leur position étant qu'ils auraient utilisé des switchs existants si ceux-ci implémentaient toutes les fonctionnalités Openflow leur étant nécessaires). Grâce à ce nouveau paradigme, Google affirme (en 2012) obtenir des performances dix fois supérieures en terme de débit, et surtout utiliser 100\%\footnote{\url{http://www.networkworld.com/article/2189197/lan-wan/google-s-software-defined-openflow-backbone-drives-wan-links-to-100--utilization.html}} de leurs lignes (contrairement aux 30 à 40\% en vigueur dans l'industrie, notamment pour garantir un service même en cas de nombreuses pannes, ce qui n'est plus nécessaire avec un réseau SDN qui adapte automatiquement le routage pour pallier aux problèmes).\\

Microsoft semble également s'être intéressé au SDN pour son service "Azure" depuis quelques années maintenant \footnote{\url{http://www.networkworld.com/article/2937396/cloud-computing/microsoft-needs-sdn-for-azure-cloud.html}}, et de manière générale toutes les figures de proue de l'industrie  numérique (celles qui disposent de nombreux serveurs et gèrent des flux énormes et grandissants de données comme Amazon, AT\&T, Facebook, ...) se sont plus ou moins annoncées investies dans le processus, possiblement avec leur propre protocole SDN. Si des grosses entreprises ont annoncé l'utilisation de ce type de réseau, les données précises concernant les performances obtenues sont difficilement accessibles, ce qui rend compliquée l'évaluation des avantages réels de SDN.\\

Par ailleurs, les switchs compatibles Openflow, ou les switchs Openflow seuls, que fournissent (depuis 2011 environ) certaines entreprises majeures du domaine comme Brocade, HP, IBM, Juniper ou encore NEC, Pronto ou Pica8, manquent encore parfois de maturité (RFC parfois interprétée différemment, vulnérabilités spécifiques, ...). L'évolution des versions Openflow est également parfois difficile à suivre (la spécification de la version 1.0.0 (fin 2009) fait 42 pages\footnote{\label{OF_10}\url{http://archive.openflow.org/documents/openflow-spec-v1.0.0.pdf}}, celle de la dernière version stable (1.5.0, fin 2014) en fait 277 \footnote{\label{OF_15}\url{https://www.opennetworking.org/images/stories/downloads/sdn-resources/onf-specifications/openflow/openflow-switch-v1.5.0.noipr.pdf}}). Or l'industrie dans ce domaine présente une certaine inertie (peu sont les compagnie qui proposent des switchs compatibles avec la dernière version d'Openflow, certaines vendant encore des switchs Openflow 1.0).\\

Les applications semblent donc d'un premier abord limitées aux datacenters (beaucoup de données à traiter, grande variabilité de la topologie (les machines virtuelles changent souvent d'emplacement/adresse)). En réalité, SDN peut être utilisé de manière effective dans plusieurs cas :

\begin{list}{$\Asteriscus$}{}
\item Réseaux d'entreprises/universités : dans le cas de nombreux équipements/protocoles différents utilisés, SDN permet théoriquement de faciliter le déploiement de politiques réseau complexes.
\item Réseaux optiques : faciliter la transition, la gestion et l'incorporation des réseaux optiques au sein du réseau actuel.
\item Infrastructure based WAN (réservé aux entreprises disposant de nombreux serveurs à travers le monde) : à la manière de Google, un moyen de connecter de grosses entités en évitant les goulots d'étranglement. Egalement un moyen envisageable pour un utilisateur de se connecter depuis n'importe quel endroit en disposant des services auxquels il souscrit.
\item Infrastructure personnelle, petites entreprises : surveiller le trafic et alerter l'administrateur local ou le fournisseur internet en cas de détection d'activité réseau suspecte (utile notamment dans le cadre de l'IoT).\\
\end{list}

Au final, une technologie pouvant sembler prometteuse, mais demeurant assez peu utilisée pour plusieurs raisons. Nous allons en étudier deux d'entre elles par la suite : la sécurité générale du réseau, et l'importance capitale du contrôleur qui devient quasiment l'unique clé de voûte du système.