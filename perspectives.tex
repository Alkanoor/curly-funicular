A l'issue de ce stage, après avoir successivement découvert le paradigme Software Defined Networking, évalué sa sécurité sur la base de ce qui avait déjà été publié sur le sujet et testé des attaques plus ou moins réalistes contre ce genre de réseau, je me suis convaincu, en partie irrationnellement (puisqu'il n'existe pas vraiment de cas d'école démontrant sur la base de mesures réelles la supériorité ou non de SDN en termes de performance) que les réseaux futurs seraient au minimum en partie des réseaux de ce type.\\
Si la sécurité complète de ces réseaux semble impossible à obtenir, puisqu'une entité reste toujours aussi faible que son point le plus faible et qu'il existe de nombreux points d'entrée sur un réseau SDN, il est nécessaire que ce soit l'une des composantes principales sur laquelle se base la construction ou l'amélioration d'un contrôleur.\\
Déployé localement et sur des réseaux isolés, les éventuelles faiblesses du paradigme peuvent ne pas s'avérer trop problèmatiques pour un administrateur conscient des risques encourus, surtout au regard des améliorations apportées en terme de flexibilité.\\
L'implantation durable de cette technologie n'est pas encore claire mais il semblerait que la tendance des investissements dans le domaine soit à une hausse significative depuis 2015 et continue de l'être. Même si la technologie n'est pas encore complètement mature, et que la spécification Openflow évolue très rapidement avec une complexité croissante (forçant une compatibilité partielle et des spécifications parfois mal appliquées au niveau des switchs), le nombre de projets et de tests semble proliférer. Certes la mise en production généralisée n'est pas pour tout de suite, mais de mon point de vue les avantages théoriques du software defined network surpassent ses inconvénients. L'important étant de rester conscient des nombreuses faiblesses d'une telle solution.\\

Aussi, c'est une ouverture optimiste que je formule : même si SDN n'est pas encore prêt pour la gestion de réseaux critiques, l'utilisation de l'ensemble contrôleur Openflow + switchs compatibles + Openflow permet de remplacer de nombreux protocoles propriétaires ou trop lourds tout en fournissant une vue détaillée et globale de l'état du réseau ainsi qu'un moyen d'appliquer facilement des politiques réseau complexes. C'est pourquoi il conviendrait à mon sens d'accompagner la transition vers ce type de réseau en fournissant petit à petit des moyens de consolidation, d'évaluation et de certification, de contrôleurs comme de switchs, principalement concernant leur sécurité. Pour en revenir une dernière fois avec l'analogie du système d'exploitation classique : qui s'imaginerait encore aujourd'hui coder en assembleur des pièces logicielles de plus en plus complexes ? Si on s'imagine assez mal à quoi pourraient ressembler des pièces logicielles réseau de plus en plus complexes, c'est peut être aussi parce qu'on a toujours considéré le réseau non pas comme quelque chose qui s'adapte en permanence à son environnement en fournissant un service dynamique, mais comme un élément statique servant uniquement à transporter de l'information d'un endroit à un autre. Il est peut être temps de changer d'approche ! (Cependant il ne faut pas se laisser distraire par les mots et les analogies, seule la réalité et les expériences de déploiement à large échelle pourront confirmer ou non les qualités supposées du SDN).\\

D’un point de vue personnel, je suis très satisfait d'avoir pu découvrir ce sujet dont je ne connaissais pas l'acronyme au départ, le tout au sein d'une équipe de recherche très conviviale. J'ai donc aussi pu me familiariser avec le monde de la recherche, ses avantages (liberté intellectuelle énorme, organisation libre, ...) et ses inconvénients (buter longtemps et en partie seul sur certains problèmes pas forcément intéressants). Encore une fois, SDN trace la voie des réseaux du futur, mais fournit un travail conséquent aux développeurs pour que le concept puisse devenir une perspective commerciale sécurisée et performante.