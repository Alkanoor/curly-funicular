L'objectif final et concret du stage s'est constitué durant les deux premiers mois, pendant lesquels j'ai étudié les réseaux SDN (ne connaissant pas le domaine), puis expérimenté sur le protocole OpenFlow avec Scapy et Wireshark avant de constituer un rapide état de l'art en matière d'attaques (générales) sur les réseaux SDN. Ensuite je me suis focalisé d'avantage sur ONOS comme l'intitulé du stage le prévoyait et j'ai réalisé diverses attaques déjà existantes pour la plupart mais sans réelles implémentations existantes.\\
Comme le mentionnait la fin de la section précédente, le contrôleur SDN est le point névralgique de toute l'infrastructure. Si on le compromet d'une manière où d'une autre, les conséquences peuvent être désastreuses : en cas de prise de contrôle à distance du contrôleur, il est possible d'espionner tout ce qui transite sur le plan de données, mais également supprimer, rediriger ou modifier le trafic. L'étude a donc eu pour but de déterminer les principaux vecteurs d'attaque envisageables dans ce genre de réseau, vecteurs ensuite appliqués sur le contrôleur ONOS.\\
Pour résumer, un audit (se voulant le plus exhaustif possible, même si il est impossible de couvrir l'ensemble des vulnérabilités d'un tel élément logiciel) a été réalisé. Cet audit a été conçu informellement à partir de la méthode STRIDE\footnote{STRIDE = Spoofing, Tampering, Repudiation, Info disclosure, Denial of service, privilege Escalation, qui respectivement permettent de vérifier les propriétés d'authentification, d'intégrité, de non-répudiation, de confidentialité, de disponibilité et d'autorisation} (créée et utilisée par microsoft à la base, cette manière de modéliser les menaces dans le domaine de la sécurité s’est beaucoup répandue). Certains scénarios ont été expérimentés pour prouver la faisabilité d'attaques précises (donc sous certaines hypothèses qui seront décrites). D'autres faiblesses sont également détaillées dans leur aspect théorique sur la base de sources externes. Bien que n'ayant pas eu accès à une situation réelle de déploiement SDN, j'essaierai de donner une conclusion pas trop biaisée aux tests effectués, et formulerai quelques recommandations et remarques.