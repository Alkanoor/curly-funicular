Mon stage, dont le sujet n'était pas complètement fixé au départ, a pris la tournure suivante : d'abord une étude des réseaux SDN (ne connaissant pas le domaine), puis début d'expérimentations sur le protocole Openflow, avec Scapy et Wireshark. Ensuite, constitution d'un rapide état de l'art en matière d'attaques (générales) sur les réseaux SDN. Puis, l'orientation s'est faite sur l'étude plus précise d'un contrôleur SDN particulier (ONOS), avec la conception de scénarios d'attaque, suivie de leur réalisation.\\
Comme on l'a dit au-dessus, le contrôleur SDN est le point névralgique de toute l'infrastructure. Si on le compromet d'une manière où d'une autre, les conséquences peuvent être désastreuses. L'étude a donc eu pour but de déterminer les principaux vecteurs d'attaque envisageables dans ce genre de réseau, principalement concernant le contrôleur ONOS (principalement, parce qu'il  est possible d'appliquer une majorité des attaques sur d'autres contrôleurs, même si cela n'a pas été expérimentalement vérifié).\\
Pour résumer, un audit (se voulant le plus exhaustif possible, même si il est impossible de couvrir l'ensemble des vulnérabilités d'un tel élément logiciel) a été réalisé. Cet audit a été conçu informellement à partir de la méthode STRIDE\footnote{STRIDE = Spoofing, Tampering, Repudiation, Info disclosure, Denial of service, privilege Escalation, qui respectivement permettent de vérifier les propriétés d'authentification, d'intégrité, de non-répudiation, de confidentialité, de disponibilité et d'autorisation} (utilisée par microsoft à la base, cette manière de modéliser les menaces dans le domaine de la sécurité s’est beaucoup répandue). Certains scénarios ont été expérimentés pour prouver la faisabilité d'attaques précises (donc sous certaines hypothèses qui sont décrites). D'autres faiblesses sont également détaillées dans leur aspect théorique sur la base de sources externes. Bien que n'ayant pas eu accès à une situation réelle de déploiement SDN, j'essaierai de donner une conclusion pas trop biaisée aux tests effectués, et formulerai quelques recommandations et remarques.