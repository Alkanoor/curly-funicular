L'idée d'une séparation entre plan de données et plan de contrôle n'est pas nouvelle, et cette partie se propose de retracer succinctement diverses voies qui ont abouti à l'adoption assez univoque du protocole OpenFlow comme interface entre les deux. Pour obtenir un état de l'art détaillé et complet, il est possible de consulter les articles "A Survey of Software-Defined Networking: Past, Present, and Future of Programmable Networks"\footnote{\label{histoire}\url{https://hal.inria.fr/hal-00825087/file/hal_final.pdf}} (2014) et "Software-Defined Networking: A Comprehensive Survey"\footnote{\url{http://www.hit.bme.hu/~jakab/edu/litr/SDN/Long_Survey_06994333.pdf}} (2015).\\

Assez tôt est apparue l'idée de rendre les switchs programmables : dès 1990, le groupe Open Signaling propose un protocole de contrôle de switchs à distance appelé GSMP (General Switch
Management  Protocol). Si les possibilités restent assez limitées (essentiellement gestion des ports, redirection de trafic sur des ports contrôlés, et obtention de statistiques), cela permet néanmoins un accès au matériel plus aisé.

D'autres approches sont testées dans la même période : l'initiative Active Networking propose quant à elle un mécanisme de propagation de code que l'équipement réseau exécute lorsqu'il reçoit les paquets encapsulant le code (même si cela pose un gros problème en matière de sécurité).

Le groupe DCAN (Devolved Control of ATM Networks) propose une approche qui se rapproche très fortement du paradigme SDN : ils développent un protocole minimaliste entre une entité spécialisée (le contrôleur) et autres équipements et mettent en place une gestion semi-automatique du réseau pour partitionner les ressources disponibles (ils rajoutent donc la possibilité de programmer le contrôleur).\\

Le projet 4D \footnote{\url{http://www.cs.cmu.edu/~4D/}} initié en 2004, présente une formalisation du concept : on cherche à obtenir la possibilité de prendre des décisions réseau en dehors des équipements physiques, ce qui nécessite l'obtention d'un maximum d'informations à la fois sur la composition physique du réseau, et sur les liens qui existent entre chaque élément. L'incorporation des services globaux que sont la découverte de la topologie et la dissémination d'informations sur l'état général du réseau associée à la possibilité d'agir sur le réseau est ce qui a inspiré l'idée de système d'exploitation réseau, qu'implémentent aujourd'hui les contrôleurs SDN.\\

On peut encore citer NETCONF et Ethane (2006), le premier pouvant être vu comme une extension de SNMP, le second comme un ancêtre immédiat d'OpenFlow. Même si la finalité d'Ethane était plus axée sur une gestion des identités (vérification des droits d'un paquet à circuler sur le réseau entre autres) que sur une gestion générale du réseau, c'est un protocole entre switch programmable et contrôleur encapsulant des actions à effectuer sur des paquets reçus au niveau du switch (ces actions étant essentiellement limitées à de la redirection/suppression).\\

OpenFlow a quant à lui précédé l'apparition du terme SDN lors d'expérimentations à Stanford vers 2010 (la première spécification d'OpenFlow pour la production (1.0.0), a été publiée début 2010).