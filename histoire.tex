L'idée d'une séparation entre plan de données et plan de contrôle n'est pas nouvelle, et cette partie se propose de retracer succinctement diverses voies qui ont abouti à l'adoption assez univoque du protocole Openflow comme interface entre les deux. Pour obtenir un état de l'art détaillé et complet, il est possible de consulter les articles "A Survey of Software-Defined Networking: Past, Present, and Future of Programmable Networks"\footnote{\label{histoire}\url{https://hal.inria.fr/hal-00825087/file/hal_final.pdf}} (2014) et "Software-Defined Networking: A Comprehensive Survey"\footnote{\url{http://www.hit.bme.hu/~jakab/edu/litr/SDN/Long_Survey_06994333.pdf}} (2015).


Assez tôt lors de l'essor d'internet tel que nous le connaissons, des idées pour fournir une sorte d'API réseau ont émergées (milieu des années 1990 environ) : le groupe Open Signaling propose un protocole d'accès universel au matériel (switchs) permettant de distribuer facilement des nouveaux services, pendant qu'Active Networking propose un mécanisme de propagation de code que l'équipement réseau exécute lorsqu'il reçoit les paquets encapsulant le code (ce qui pose au passage un énorme problème de sécurité).\\

Dans le même temps, le groupe DCAN (Devolved Control of ATM Networks) propose une approche qui se rapproche très fortement du paradigme SDN : convaincus que les fonctions de contrôle et de gestion des différents éléments du réseau doivent être séparées du routage des données et déléguées à des équipements spécialisés, ils développent un protocole minimaliste entre contrôleur et autres équipements, à la manière du protocole Openflow aujourd'hui majoritaire dans les réseaux SDN.\\

Le projet 4D \footnote{\url{http://www.cs.cmu.edu/~4D/}} initié en 2004 (et semblant s'être fini un peu avant 2010), ajoute quant à lui des abstractions diverses (découverte des voisins proches et remontée des informations, dissémination d'informations sur l'état général du réseau, puis prise de décision sur la base des informations récoltées). C'est ce genre de projet qui a inspiré l'idée de système d'exploitation réseau, qu'implémentent les contrôleurs SDN.\\

On peut encore citer NETCONF et Ethane (2006), le premier pouvant être vu comme une extension de SNMP, le second comme un ancêtre immédiat d'openflow (un contrôleur qui décide si et où un paquet devrait être redirigé, et des switchs qui constitués d'une table de flux et d'un canal sécurisé vers le contrôleur).\\

Openflow a quant à lui précédé l'apparition du terme SDN lors d'expérimentations à Stanford vers 2010 (la première spécification d'Openflow pour la production (1.0.0), a été publiée début 2010).