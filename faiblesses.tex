Pour tester le contrôleur, l’installation suivante est actuellement réalisée :
- une machine virtuelle utilisant l'émulateur réseau mininet, qui crée des switchs et hôtes virtuels, et qui permet de générer du trafic réseau SDN.
- une machine virtuelle (ubuntu server) sur laquelle le contrôleur ONOS est installé et est accessible.
- une machine virtuelle (ubuntu desktop) permettant d’intéragir avec le contrôleur en SSH (on peut aussi utiliser la machine non virtuelle).

On va donc appliquer la méthode STRIDE aux divers éléments et interfaces qui composent ONOS, à la manière de ce qui est présenté dans un article de l'institut Fraunhofer \footnote{\url{http://publica.fraunhofer.de/eprints/urn_nbn_de_0011-n-4046948.pdf}}.
