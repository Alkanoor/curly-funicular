\underline{Prérequis/Hypothèses :}\\
- Un éditeur d’application malveillant\\


\underline{Buts :}\\
Voir quelles informations sensibles il est possible de collecter sur les autres applications tournant sur le contrôleur à partir d’une application malveillante. Avec le secure mode activé ou sans.

\underline{Déroulement :}\\
Un utilisateur écrit une application d’apparence quelconque mais cherche à utiliser ce qui est à sa disposition dans l’API d’ONOS pour d’une part obtenir des renseignements sur les applications tournant à côté de notre application malveillante, et d’autre part à modifier son fonctionnement, en altérant ce qu’elle est susceptible de recevoir. Lorsque le secure mode est activé, on vérifie que l’application n’a pas accès à des fonctions critiques, et on regarde quelles informations peuvent toutefois fuiter.

\underline{Détails techniques :}\\
Se référer au scénario 5 décrit dans l'annexe (page).

\underline{Résultat :}\\
Lorsque le Security Mode n'est pas activé, ayant accès au service gérant les applications et aux services internes du contrôleur, on peut donc tout faire sur celles-ci (désactivation, envoi de données falsifiées, ...). Sinon, selon les permissions, on peut effectuer certaines actions qui ont plus ou moins d'impacts. Par exemple avec les droits d'accès en lecture au système de fichier, on peut lire certains bouts de mémoire du contrôleur et accéder à des informations pas forcément dénuées d'intérêt (notamment des chemins absolus vers certains emplacements comme le répertoire contenant les applications du contrôleur). Avec les droits d'accès aux informations des applications en lecture, on peut lister les applications présentes et obtenir d'autres informations.

\underline{Parades proposées :}\\
Si il est bien utilisé, le Security Mode est efficace pour empêcher des fuites d'information non désirées. Là encore, la responsabilité de donner des droits corrects incombe à l'administrateur et ne doit pas être négligée. Si les permissions d'une application sont réduites au minimum, celle-ci n'a plus beaucoup de possibilités. Une autre protection envisageable permettant d'isoler chaque application des applications voisines est celle qui a été implémentée dans le contrôleur Rosemary, à savoir une séparation des droits d'accès à la mémoire du contrôleur en fonction de l'application (ainsi, contrairement à ONOS au sein duquel il est possible d'accéder à toute la mémoire utilisée par le contrôleur, Rosemary interdit à une application d'accéder à des pages mémoires dont elle n'est pas à l'origine).

\underline{Limitations/Impact/probabilité :}\\
Comme précédemment, le risque est élevé. Même si l’impact est plus faible que dans la situation précédente, si il est possible d’extraire de l’information de "vraies" applications, il est envisageable que cela puisse servir en vue d’une attaque ultérieure cette fois sur les vraies applications à un niveau plus haut (en ciblant par exemple des applications avec un niveau de privilège élevé). Cela demeure toutefois assez complexe à mettre en œuvre.
